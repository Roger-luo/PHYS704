\documentclass{article}
\usepackage{amsmath}
\usepackage[utf8]{inputenc}

\title{PHYS704: Assignment 4}
\author{Xiuzhe Luo}
\date{}

\begin{document}

\maketitle

\section*{Rotating gas}
\subsection*{1.a}

$$
\begin{aligned}
    \{\vec{L}_i, \mathcal{H}\} &= \sum_k \{\vec{L}_i, T_k + U_k\} \\
    &= \sum_k \{\vec{L}_i, \frac{p^2_k}{2m}\} + \{\vec{L}_i, K\frac{r_k^2}{2}\}
\end{aligned}
$$

since there is no interaction, if $i \neq k$, $\{L_i, T_k\}, \{L_i, U_k\} = 0$, when $i = k$, we will show
$\{L, \frac{p^2}{2m}\}$ and $\{L, K\frac{r^2}{2}\}$ are zero, by definition we have

$$
\begin{aligned}
    \mathbf{L} &= \sum q_j p_k \epsilon_{ijk}\mathbf{e}_i\\
    \vec{r} &= \{q_{1}, q_{2}, q_{3}\}\\
\end{aligned}
$$

because $q_j p_k$ is symmetric, swapping $i, j$ does not effect the summation, but it change
the sign of $\epsilon$ , thus

$$
\begin{aligned}
    \{L, \frac{p^2}{2m}\} &= \sum \{q_j p_k, \frac{p^2}{2m}\} \epsilon_{ijk} \mathbf{e}_i\\
    &= \frac{1}{2}(\sum\{q_j p_k, \frac{p^2}{2m}\} \epsilon_{ijk} \mathbf{e}_i + \sum\{q_k p_j, \frac{p^2}{2m}\} \epsilon_{ikj} \mathbf{e}_i)\\
    &= \frac{1}{2}\sum(\{q_j p_k, \frac{p^2}{2m}\} \epsilon_{ijk} - \{q_j p_k, \frac{p^2}{2m}\} \epsilon_{ijk}) \mathbf{e}_i\\
    &= 0
\end{aligned}
$$

similarly, since $r^2 = q^2 = \sum_i q^2_i$, we can do the exactly same thing, and get $\{L, K\frac{q^2}{2}\} = 0$, thus $\{\mathbf{L}_i, \mathcal{H}\} = 0$,
it is also pretty straight forward to prove interacting terms in the same way, since interacting term $M_{ij} = -M_{ji}$, we have
$\{\mathbf{L}, \mathcal{H}\} = \sum_i \{L_i, \mathcal{H}\} = \sum_i \{L_i, M_{jk}\}$ become zero.

\subsection*{1.b}

since $\vec{\Omega} = \Omega \hat{z}$, $\vec{\Omega}\cdot \vec{L} = \Omega L_z = \Omega (xp_y - yp_z)$, thus

$$
\begin{aligned}
    & Z = \frac{1}{N!}(\int \frac{dq^3 dp^3}{h^3} \exp[-\beta\frac{p_x^2 + p_y^2 + p_z^2}{2m} - \frac{K\beta}{2}(x^2 + y^2 + z^2) - \beta \Omega (xp_y - yp_z)])^N\\
    &= \frac{1}{N!h^{3N}} (\frac{2\pi}{\beta}\sqrt{\frac{m}{K}}\int dx dy dp_z dp_y \exp[-\frac{\beta}{2m}(p_y^2 + p_z^2) - \frac{K\beta}{2} (x^2 + y^2) - \beta \Omega (xp_y - yp_z) ])^N
\end{aligned}
$$

and we have

$$
\begin{aligned}
    &\int dx dy dp_z dp_y \exp[-\frac{\beta}{2m}(p_y^2 + p_z^2) - \frac{K\beta}{2} (x^2 + y^2) - \beta \Omega (xp_y - yp_z) ]\\
    &=\int dy dp_z dp_y \exp[-\frac{\beta}{2m}(p_y^2 + p_z^2) - \frac{K\beta}{2} y^2 + \beta \Omega yp_z] \int dx \exp[- \frac{K\beta}{2} x^2 - \beta \Omega xp_y]\\
    &= \sqrt{\frac{2\pi}{K\beta}}\int dy dp_z dp_y \exp[-\frac{\beta}{2m}(p_y^2 + p_z^2) - \frac{K\beta}{2} y^2 + \beta \Omega yp_z + \frac{\beta \Omega^2 p_y^2}{2K}]\\
    &= \sqrt{\frac{2\pi}{K\beta}}\int dp_z dp_y \exp[-\frac{\beta}{2m}(p_y^2 + p_z^2) + \frac{\beta \Omega^2 p_y^2}{2K}] \int dy \exp[- \frac{K\beta}{2} y^2 + \beta \Omega yp_z]\\
    &= \frac{2\pi}{K\beta}\int dp_z dp_y \exp[-(\frac{\beta}{2m} - \frac{\beta \Omega^2}{2K})(p_y^2 + p_z^2)]\\
    &= \frac{2\pi}{K\beta} \frac{2\pi m K}{K\beta - m\beta \Omega^2} = \frac{4\pi^2 m}{\beta (K\beta - m\beta \Omega^2)}
\end{aligned}
$$

the last integral uses the assumption that $\Omega < \sqrt{K/m}$, so the partition function $Z$ is

$$
\begin{aligned}
    Z = \frac{1}{N!h^{3N}}(\sqrt{\frac{m}{K}} \frac{8\pi^3 m}{\beta^3 (K - m \Omega^2)})^N = \frac{1}{N!}[(\frac{2\pi}{\beta h})^3\sqrt{\frac{m}{K}} \frac{m}{K - m \Omega^2}]^N
\end{aligned}
$$

\subsection*{1.c}

$$
\begin{aligned}
    &\langle L_z \rangle = \frac{1}{N!Z}\int (\prod_{i=1}^N \frac{d^3q_i d^3p_i}{h^3}) L_z \exp[-\beta \mathcal{H(\mu)} - \beta \Omega L_z]\\
    &= -\frac{1}{N!Z}\int (\prod_{i=1}^N \frac{d^3q_i d^3p_i}{h^3}) \frac{1}{\beta} \frac{\partial}{\partial \Omega} \exp[-\beta \mathcal{H(\mu)} - \beta \Omega L_z]\\
    &= -\frac{1}{\beta Z} \frac{\partial}{\partial \Omega} \frac{1}{N!} \int (\prod_{i=1}^N \frac{d^3q_i d^3p_i}{h^3}) \exp[-\beta \mathcal{H(\mu)} - \beta \Omega L_z]\\
    &= -\frac{1}{\beta Z} \frac{\partial Z}{\partial \Omega}\\
\end{aligned}
$$

and the derivative is

$$
\begin{aligned}
    &\frac{\partial Z}{\partial \Omega} = \frac{1}{(N-1)!} [(\frac{2\pi}{\beta h})^3 \sqrt{\frac{m}{K}} \frac{m}{K - m\Omega^2}]^{N-1}
        (\frac{2\pi}{\beta h})^3 \sqrt{\frac{m}{K}} \frac{\partial}{\partial \Omega} \frac{m}{K - m\Omega^2}\\
    &=\frac{1}{(N-1)!} [(\frac{2\pi}{\beta h})^3 \sqrt{\frac{m}{K}} \frac{m}{K - m\Omega^2}]^{N-1}
    (\frac{2\pi}{\beta h})^3 \sqrt{\frac{m}{K}} \frac{m}{(K - m\Omega^2)^2} 2m\Omega\\
    &= N Z \frac{2m\Omega}{K - m\Omega^2}
\end{aligned}
$$

thus

$$
\langle L_z \rangle = -\frac{N}{\beta} \frac{2m\Omega}{K - m\Omega^2}
$$

\subsection*{1.d}

$$
\begin{aligned}
    &\rho(x, y, z, p_x, p_y, p_z) = \frac{1}{N} \frac{1}{(N-1)!} \int \prod_{i=2}^N \frac{d^3q_i d^3p_i}{h^3} \rho(\mu)\\
    &= \frac{1}{N! h^{3(N-1)}} \frac{1}{Z} [(\frac{2\pi}{\beta})^3\sqrt{\frac{m}{K}} \frac{m}{K - m \Omega^2}]^{N-1} \exp[-\beta (T_i + U_i)]\\
    &= \frac{N!h^{3N}}{N! h^{3(N-1)}} [(\frac{2\pi}{\beta})^3\sqrt{\frac{m}{K}} \frac{m}{K - m \Omega^2}]^{-1} \exp[-\beta (T_i + U_i)]\\
    &= h^{3}[(\frac{2\pi}{\beta})^3\sqrt{\frac{m}{K}} \frac{m}{K - m \Omega^2}]^{-1} \exp[-\beta (T_i + U_i)]\\
\end{aligned}
$$

$$
\begin{aligned}
    \rho(x, y, z) &= h^{3}[(\frac{2\pi}{\beta})^3\sqrt{\frac{m}{K}} \frac{m}{K - m \Omega^2}]^{-1}\exp[- \frac{K\beta}{2}(x^2 + y^2 + z^2)] \\
    &\int \frac{d^3p}{h^3} \exp[-\beta\frac{p_x^2 + p_y^2 + p_z^2}{2m} - \beta \Omega (xp_y - yp_z)]\\
\end{aligned}
$$

for the integral, we have

$$
\begin{aligned}
    &\int d^3p \exp[-\beta\frac{p_x^2 + p_y^2 + p_z^2}{2m} - \beta \Omega (xp_y - yp_z)]\\
    &= \sqrt{\frac{2\pi m}{\beta}} \int dp_y dp_z \exp[-\frac{\beta}{2m} (p_y^2 + p_z^2) - \beta \Omega x p_y + \beta \Omega yp_z]\\
    &= (\frac{2\pi m}{\beta})^{3/2} exp[\frac{1}{2}\beta m \Omega^2(x^2 + y^2)]
\end{aligned}
$$

thus

$$
\begin{aligned}
    \rho(x, y, z) &= [(\frac{2\pi}{\beta})^3\sqrt{\frac{m}{K}} \frac{m}{K - m \Omega^2}]^{-1} (\frac{2\pi m}{\beta})^{3/2}\\
    & \exp[- \frac{K\beta}{2}(x^2 + y^2 + z^2) + \frac{1}{2}\beta m \Omega^2(x^2 + y^2)]\\
    &= ((\frac{2\pi}{\beta})^{3/2}\sqrt{\frac{1}{K}} \frac{1}{K - m \Omega^2})^{-1}
    \exp[- \frac{K\beta}{2}(x^2 + y^2 + z^2) + \frac{1}{2}\beta m \Omega^2(x^2 + y^2)]
\end{aligned}
$$

we first calculate $\langle z^2 \rangle$, since

$$
\begin{aligned}
    &\int dxdydz z^2 \exp[- \frac{K\beta}{2}(x^2 + y^2 + z^2) + \frac{1}{2}\beta m \Omega^2(x^2 + y^2)]\\
    &= \frac{2\pi}{\beta (K - m\Omega^2)} \int z^2 \exp[- \frac{K\beta}{2} z^2] dz\\
    &= \frac{2\pi}{\beta (K - m\Omega^2)} \sqrt{\frac{2\pi}{K\beta}} \frac{1}{K\beta}
\end{aligned}
$$

$$
\begin{aligned}
    \langle z^2\rangle &= ((\frac{2\pi}{\beta})^{3/2}\sqrt{\frac{1}{K}} \frac{1}{K - m \Omega^2})^{-1} \frac{2\pi}{\beta (K - m\Omega^2)} \sqrt{\frac{2\pi}{K\beta}} \frac{1}{K\beta}\\
    &=\frac{1}{\beta K}\\
\end{aligned}
$$

$\langle x^2 \rangle$ is equal to $\langle y^2 \rangle$, and since

$$
\begin{aligned}
    &\int dxdydz x^2 \exp[- \frac{K\beta}{2}(x^2 + y^2 + z^2) + \frac{1}{2}\beta m \Omega^2(x^2 + y^2)]\\
    &=\sqrt{\frac{2\pi}{K\beta}} \sqrt{\frac{2\pi}{\beta (K - m\Omega^2)}} \int dx x^2 \exp[- \frac{K\beta}{2}x^2 + \frac{1}{2}\beta m \Omega^2 x^2]\\
    &=\sqrt{\frac{2\pi}{K\beta}} \frac{2\pi}{\beta (K - m\Omega^2)} \frac{1}{\beta (K - m\Omega^2)}
\end{aligned}
$$

we have their expectation

$$
\begin{aligned}
    &\langle x^2 \rangle = \langle y^2 \rangle = \sqrt{\frac{2\pi}{K\beta}} \frac{2\pi}{\beta (K - m\Omega^2)} \frac{1}{\beta (K - m\Omega^2)}
    ((\frac{2\pi}{\beta})^{3/2}\sqrt{\frac{1}{K}} \frac{1}{K - m \Omega^2})^{-1}\\
    &=\frac{1}{\beta (K - m\Omega^2)}\\
\end{aligned}
$$

\section*{Atomic/molecular hydrogen}
\subsection*{2.a}
since interaction and quantum degeneracies can be ignored, this is an ideal gas, we can use the equation from textbook,

$$
\begin{aligned}
    Z_a(N_1, T, V) &= \int \frac{1}{N_1!} \prod_{i=1}^{N_1}  \frac{d^3q_i d^3p_i}{h^3} \exp[-\beta \sum_{i=1}^{N_1} \frac{p_i^2}{2m}]\\
    &= \frac{V^{N_1}}{N_1!} (\frac{2\pi m k_B T}{h^2})^{3N_1/2}
\end{aligned}
$$

\subsection*{2.b}

since transition and rotation is independent, we have

$$
\begin{aligned}
    &Z_m(N_2, T, V) = Z_{trans} Z_{rot} \exp[N_2\beta\epsilon]\\
\end{aligned}
$$

for transition we have

$$
\begin{aligned}
    &Z_{trans} = \frac{1}{N_2!}\int \prod_{i=1}^{N_2} \frac{d^3q_i d^3p_i}{h^3} \exp[-\beta\sum_{i=1}^{N_1}\frac{p_i^2}{4m}]\\
    &= \frac{V^{N_2}}{N_2!} (\frac{4\pi m k_B T}{h^2})^{3N_2/2}
\end{aligned}
$$

for rotation, by definition of $L$, the Hamiltonian can be written as

$$
H_{rot} = \frac{p_{\theta}^2}{2I} + \frac{p_{\phi}^2}{2I\sin^2\theta}
$$

thus the partition function of rotation is

$$
\begin{aligned}
    &Z_{rot} = (\int \frac{d\theta d\phi dp_{\theta} dp_{\phi}}{h^2} \exp[-\beta (\frac{p_{\theta}^2}{2I} + \frac{p_{\phi}^2}{2I\sin^2\theta})])^{N_2}\\
    &= (\frac{1}{h^2} \sqrt{\frac{2\pi I}{\beta}} \int_{0}^{\pi} d\theta \sqrt{\frac{2\pi I \sin^2\theta}{\beta}} \int_{0}^{2\pi} d\phi)^{N_2}\\
    &= (\frac{8\pi^2 I}{\beta h^2})^{N_2}\\
    &= (\frac{8\pi^2 I k_B T}{h^2})^{N_2}\\
\end{aligned}
$$

thus the partition function is

$$
\begin{aligned}
    Z &= \frac{1}{N_2!} (V(\frac{4\pi m k_B T}{h^2})^{3/2} (\frac{8\pi^2 I k_B T}{h^2}) \exp[\beta\epsilon])^{N_2}\\
    &= \frac{1}{N_2!} (\frac{16\sqrt{2}\pi^2 I k_B T V}{\lambda^3 h^2} \exp[\beta\epsilon])^{N_2}
\end{aligned}
$$

where $\lambda = \frac{h}{\sqrt{2\pi mk_B T}}$

\subsection*{2.c}

$$
F_1 = -k_B T \ln(Z) = -N_1 k_B T [\ln(\frac{Ve}{N_1}) + \frac{3}{2}\ln(\frac{2\pi mk_B T}{h^2})]
$$

$$
\begin{aligned}
    F_2 &= -k_B T \ln(Z) =-k_B T \ln(Z)\\
    &= -k_B T \ln(\frac{1}{N_2!} (\frac{16\sqrt{2}\pi^2 I k_B T V}{\lambda^3 h^2} \exp[\beta\epsilon])^{N_2})\\
    &= -k_B T N_2 (-\ln(N_2) + 1 + \ln(\frac{16\sqrt{2}\pi^2 I k_B T V}{\lambda^3 h^2} \exp[\beta\epsilon]))
\end{aligned}
$$

when equilibrium, the Gibbs free energy is a constant, indicates

$$
dG = \mu_a dN_a + \mu_m dN_m = 0, 2dN_m = -dN_a
$$


thus $2\mu_a = \mu_m$, 

$$
\begin{aligned}
    \mu_a = \left.\frac{\partial F_1}{\partial N_1}\right|_{T, V} = -k_B T \ln(\frac{V}{N_1 \lambda^3})
\end{aligned}
$$

$$
\begin{aligned}
    \mu_m &= \left.\frac{\partial F_2}{\partial N_2}\right|_{T, V}\\
    &= -k_B T (-\ln(N_2) + \ln(\frac{16\sqrt{2}\pi^2 I k_B T V}{\lambda^3 h^2} \exp[\beta\epsilon]))\\
    &= -k_B T \ln(\frac{16\sqrt{2}\pi^2 I k_B T V}{\lambda^3 h^2 N_2} \exp[\beta\epsilon])
\end{aligned}
$$

so we have

$$
\begin{aligned}
    (\frac{V}{N_1 \lambda^3})^2 &= (\frac{16\sqrt{2}\pi^2 I k_B T V}{\lambda^3 h^2 N_2} \exp[\beta\epsilon])\\
    (\frac{1}{n_a \lambda^3})^2 &= (\frac{16\sqrt{2}\pi^2 I k_B T}{\lambda^3 h^2 n_m} \exp[\beta\epsilon])\\
    \frac{n_m}{n_a^2} &= (\frac{16\sqrt{2}\pi^2 \lambda^3 I k_B T}{h^2} \exp[\beta\epsilon])\\
\end{aligned}
$$

\section*{Molecular adsorption}

\subsection*{3.a}

the smallest energy is zero, and since it has to be aligned in $x$ or $y$ directions, thus we have

$$
\Omega(0, N) = \frac{N!}{(N-2)! N!} = N(N-1)
$$

the largest energy is when they all stand up, which is $N\epsilon$

\subsection*{3.b}

at energy $E$, the number of zero energy molecules are $N - N_1$, they can be $x$ or $y$ direction,
the number of macrostates is

$$
\Omega(E, N) = \frac{N!}{N_1!(N-N_1)!} (N-N_1)(N-N_1-1) = \frac{N!}{N_1!(N-N_1-2)!}
$$

where $N_1 = \frac{E}{\epsilon}$, assume $N_1, N >> 1$, the entropy is

$$
\begin{aligned}
    S(E, N) &\approx -Nk_B [\frac{N_1}{N}\ln\frac{N_1}{N} + \frac{N-N1-2}{N}\ln\frac{N-N_1-2}{N}]\\
    &= -Nk_B[\frac{E}{N\epsilon}\ln(\frac{E}{N\epsilon}) + (1 - \frac{E}{N\epsilon}- \frac{2}{N})\ln(1 - \frac{E}{N\epsilon}- \frac{2}{N})]
\end{aligned}
$$

\subsection*{3.c}
The energy can be calculated using

$$
\frac{1}{T} = \left.\frac{\partial S}{\partial E} \right|_N = -\frac{k_B}{\epsilon} \ln(\frac{E}{N\epsilon - E - 2\epsilon})
$$

thus we have the internal energy

$$
E(T) = \frac{(N-2)\epsilon}{\exp(\frac{\epsilon}{k_B T}) + 1}
$$

the heat capacity is

$$
\begin{aligned}
    C(T) = \frac{dE}{dT} = (N-2)k_B (\frac{\epsilon}{k_B T})^2 \exp(\frac{\epsilon}{k_B T})[\exp(\frac{\epsilon}{k_B T}) + 1]^{-2}
\end{aligned}
$$

% TODO: sketch

\subsection*{3.d}

$$
\begin{aligned}
    p(n=0_x) = p(n=0_y) &= \frac{\Omega(E, N-1)}{\Omega(E, N)}\\
    &= \frac{(N-1)!}{N_1!(N-N_1-3)!} \frac{N_1!(N-N_1-2)!}{N!}\\
    &= (1-\frac{N_1}{N}-\frac{2}{N})
\end{aligned}
$$

thus the probability of standing up is

$$
\begin{aligned}
    & p(n=1) = 1 - 2p(n=0_x) \\
    &= 1 - 2(1 - \frac{N_1}{N} - \frac{2}{N})\\
    &= \frac{2N_1}{N} + \frac{4}{N} - 1\\
    &= \frac{2E}{N\epsilon} + \frac{4}{N} - 1\\
    &= \frac{2(1-2/N)}{\exp(\frac{\epsilon}{k_BT}) + 1} + \frac{4}{N} - 1
\end{aligned}
$$

\subsection*{3.e}

this is when $T \rightarrow +\infty$, which gives $E = (N-2)\epsilon$


\section*{Curie susceptibility}
\subsection*{4.a}

$$
\begin{aligned}
    Z(T, B) &= \sum_{\{m_i\}} \exp[\beta B\mu \sum_{i=1}^{N} m_i]\\
    &= \prod_{i=1}^N (\sum_{\{m_i\}}\exp[\beta B\mu m_i])\\
    &= (\exp[-\beta B\mu s] \sum_{m = 1}^{2s+1} \exp[\beta B\mu m])^N\\
    &= (\frac{\exp[-\beta B\mu s]- \exp[\beta B\mu (s+1)]}{1 - \exp[\beta B\mu]})^N\\
    &= (\frac{\exp[-\beta B\mu (s + 1/2)]- \exp[\beta B\mu (s+1/2)]}{\exp[-\beta B\mu/2] - \exp[\beta B\mu/2]})^N\\
    &= (\frac{\sinh(\beta B\mu[s + 1/2])}{\sinh(\beta B\mu/2)})^N
\end{aligned}
$$

\subsection*{4.b}

Gibbs free energy

$$
\begin{aligned}
    G &= -k_B T \ln Z\\
    &= -N k_B T \ln(\frac{\sinh(\beta B\mu[s + 1/2])}{\sinh(\beta B\mu/2)})\\
    &= -N k_B T [\ln(\sinh(\beta B\mu[s + 1/2])) - \ln(\sinh(\beta B\mu/2))]\\
    &\approx -N k_B T [\ln(\beta B\mu[s + 1/2] + \frac{(\beta B\mu[s + 1/2])^3}{6}) - \ln(\beta B\mu/2 + \frac{(\beta B\mu/2)^3}{6})]\\
    &= -N k_B T [\ln(\beta B\mu (s+1/2)) + \ln(1 + \frac{(\beta B\mu[s + 1/2])^2}{6}) - \ln(\beta B\mu/2) - \ln(1 + \frac{(\beta B\mu/2)^2}{6})]\\
    &= G(0) -N k_B T [\ln(1 + \frac{(\beta B\mu[s + 1/2])^2}{6}) - \ln(1 + \frac{(\beta B\mu/2)^2}{6})]\\
    &= G(0) -N k_B T [\frac{(\beta B\mu[s + 1/2])^2}{6} - \frac{(\beta B\mu/2)^2}{6}]\\
    &= G(0) - \frac{N\mu^2 s(s+1) B^2}{6k_B T}
\end{aligned}
$$

since $G(B)$ is even, thus the high order terms start from $O(B^4)$

\subsection*{4.c}

$$
M = -\frac{\partial G}{\partial B} = \frac{2N\mu^2 s(s+1)}{3k_B T} B + O(B^3)
$$

thus

$$
\chi = \left.\frac{\partial M}{\partial B}\right|_{B=0} = \frac{2N\mu^2 s(s+1)}{3k_B T} = c/T
$$

\subsection*{4.d}

$$
\begin{aligned}
    C_B &= \left.\frac{\partial H}{\partial T}\right|_B\\
    &= -B \left.\frac{\partial M}{\partial T}\right|_B\\
    &= -B \frac{\partial}{\partial T} \frac{2N\mu^2 s(s+1)}{3k_B T} B\\
    &= B^2 \frac{2N\mu^2 s(s+1)}{3k_B T^2}\\
\end{aligned}
$$

$$
C_M = \left.\frac{\partial H}{\partial T}\right|_M = -M \left.\frac{\partial B}{\partial T}\right|_M = 0
$$

thus $C_B - C_M = cB^2/T^2$

\section*{Disordered glass}
\subsection*{6.a}

it is a two level system

$$
\begin{aligned}
    Z(T) &= \sum_{m_i \in {0, 1}} \exp[-\beta \sum_i \epsilon_i + \delta_i m_i]\\
    &= \prod_{i=1}^N \exp[\beta \epsilon_i] \sum_{m_i \in 0, 1} \exp[-\beta\delta_i m_i]
\end{aligned}
$$

\end{document}